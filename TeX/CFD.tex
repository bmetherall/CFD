\documentclass[10pt, titlepage]{article}
\usepackage[english]{babel}
\usepackage{amsmath}
\usepackage{amsfonts}
\usepackage{xcolor,graphicx}
\usepackage{geometry}
\usepackage{animate}
\usepackage{listings}
\usepackage{wrapfig}
\usepackage{subcaption}
\usepackage{hyperref}

\usepackage{lipsum}

% http://tex.stackexchange.com/questions/258501/how-to-make-a-combined-list-of-figures-and-tables
% Based on `tocloft` package
\usepackage[titles]{tocloft}
\newlistof{figtab}{loft}{List of Figures and Tables}
\makeatletter
% Change the file extension of both lot and lof
\def\ext@figure{loft}
\def\ext@table{loft}
% Store the original `\thefigure` and `\thetable`
\let\tohe@thefigure\thefigure
\let\tohe@thetable\thetable
% Redefine them to contain a "dummy" `\tohe@list...`
\def\thefigure{\tohe@listfig\tohe@thefigure}
\def\thetable{\tohe@listtab\tohe@thetable}
% Make the two dummy commands truly dummy
\let\tohe@listfig\relax
\let\tohe@listtab\relax
% Store the original `\listoffigtab`
\let\tohe@listoffigtab\listoffigtab
% Redefine it in such a way that the dummy commands insert "Fig." or "Tab." respectively
\def\listoffigtab{%
  \begingroup
  \def\tohe@listfig{Fig.~}
  \def\tohe@listtab{Tab.~}
  \tohe@listoffigtab
  \endgroup  
}
% Change \listoffigtab spacing
\setlength{\cftfignumwidth}{5em}
\setlength{\cfttabnumwidth}{5em}
\setlength{\cftfigindent}{0pt}
\setlength{\cfttabindent}{0pt}
\makeatother

%Modified from:
%http://mathematica.stackexchange.com/a/83811

\definecolor{listinggray}{gray}{0.9}
\definecolor{graphgray}{gray}{0.7}
\definecolor{blue}{rgb}{0,0,1}
\definecolor{mygreen}{rgb}{0,0.6,0}
\definecolor{mygray}{rgb}{0.5,0.5,0.5}
\definecolor{purple}{rgb}{0.5,0,0.5}
\definecolor{orange}{rgb}{1,0.5,0}

% define a custom mathematica language for syntax highlighting
\lstdefinelanguage{myMMA}{
basicstyle=\ttfamily,
keywords={SetDirectory, ColorData, BarLegend, LegendLabel, Graphics, Gray, Polygon, Import, Do, If, Dimensions, Break, AppendTo, Sqrt, All, Length, ListDensityPlot, ColorFunction, PlotRange, AspectRatio, PlotLegends, ColorFunctionScaling, FrameLabel, PlotLabel, Style, FontSize, Black, ImageSize, Full, Show, Automatic, False, LegendMarkerSize, ToString, $CommandLine, Print, LabelStyle, ListContourPlot, ContourShading, Export, Contours, None, Frame, True},
keywords=[2]{i},
keywordstyle=\color{black},
keywordstyle=[2]\color{mygreen},
commentstyle=\color{gray}, 
stringstyle=\color{purple},
identifierstyle=\color{blue},
sensitive=false,
comment=[l]{(*},
morecomment=[s]{/*}{*/},
morecomment=[l][\color{orange}]{\#!},
morestring=[b]',
morestring=[b]",
breaklines=true,
captionpos=b,
numbers=left,
literate={->}{$\rightarrow{}$}{1}
}

\title{Computational Fluid Dynamics \\ \large using the SU$^2$ Code}
\author{Brady Metherall \and 100516905}
\date{Monday April 4, 2016}

\begin{document}
\newgeometry{margin=1in}
\maketitle
\setlength\parindent{0pt}
\lstset{language=myMMA}

\listoffigtab

\section{Introduction}

\begin{wrapfigure}{R}{0.45\textwidth}
\centering
\includegraphics[width = 0.45\textwidth]{Pictures/aero}
\caption[Example of Computational Fluid Dynamics Applications]{Computational fluid dynamics has a wide range of applications -- passenger aircraft for instance \cite{aeroplane}.}
\label{fig:aeroplane}
\end{wrapfigure}

Before the invention of high speed computers, and GPU computing, fluid dynamics was a two discipline field -- theoretical, and experimental. However, over the past few decades, a new discipline has emerged, computational fluid dynamics. Computational fluid dynamics has grown to an equal footing with the other two sub-fields, although, it will never be a replacement.  Computational fluid dynamics is essentially a perfect wind tunnel; both give important information about the fluid flow around a shape such as an airplane, both allow parameters such as the Reynolds number, temperature, or flow speed to be changed, and both are an effective way to optimize the aerodynamics of an object. Where computational fluid dynamics pulls ahead of wind tunnels is its cost, speed, ease of use, and range of operation. Changing the Reynolds number, or temperature in a computational fluid dynamics experiment is as easy as typing in a new number; it's not quite this simple in a wind tunnel. Furthermore, computational fluid dynamics has the ability to provide a broader range of data, as well as simulate impractical situations to test in a wind tunnel, such as the extreme speeds and temperatures a sub-orbital spacecraft would experience. Computational fluid dynamics experiments can even be done from anywhere in the world, by anyone via remote access. \\

Computational fluid dynamics has a wide variety of applications even today. There are some things that only computational fluid dynamics simulations can predict, such as how molten iron will flow into a mold in a manufacturing plant. Computational fluid dynamics plays an important role in other areas as well including improving fuel efficiency of automobiles, optimizing pumps and turbines in dams, and even testing naval submarines and torpedoes. Computational fluid dynamics rivals other areas of physics and engineering as a vital area of research. \\

In this project we run basic computational fluid dynamics simulations for two dimensional flows. The Stanford University Unstructured (SU$^2$) code was used to run both laminar and turbulent simulations. We first examine some background theory, and numerical simplifications; our first experiment looks at a classic airfoil. Then, a static cylinder using both Euler's equations as well as the Navier-Stokes equations was simulated. Finally, we experimented with turbulent flow past a static cylinder and observed vortex shedding and von K\'{a}rm\'{a}n vortex streets.

\newpage

\section{Theory}
\subsection{Non-Turbulent Flow Equations}

Non-viscous, ideal fluids obey two equations of motion collectively called Euler's equations. The first is the incompressibility condition 
\begin{align}
\nabla \cdot \mathbf{u} &= 0,
\label{eq:incomp}
\end{align}
which arises by assuming the volume of a fluid element, as well as its density are constant. The second equation,
\begin{align}
\frac{\partial \mathbf{u}}{\partial t} + \left( \mathbf{u} \cdot \nabla \right) \mathbf{u} &= - \frac{1}{\rho} \nabla p + \mathbf{g}, \label{eq:euler}
\end{align}
can be derived from Newton's second law. The problem with Euler's equations is that they are only applicable for non-viscous fluids, which is not very practical. One of the subtler details is that the force experienced by a fluid element is always normal to its surface. However, due to the fluid elements interacting, tangential forces arise. Cauchy's equation,
\begin{align}
\rho \left( \frac{\partial \mathbf{u}}{\partial t} + \left( \mathbf{u} \cdot \nabla \right) \mathbf{u} \right) &= \nabla \cdot \mathbf{T} + \rho \mathbf{g},
\label{eq:cauchy}
\end{align}
adds a stress tensor to handle these additional forces. If we chose $T_{ij} = -p \delta_{ij}$ we simply get \eqref{eq:euler}, as expected, however, if we chose a more complex tensor with a deformation term, such as
\begin{align*}
T_{ij} = -p \delta_{ij} + \mu \left( \frac{\partial u_j}{\partial x_i} + \frac{\partial u_i}{\partial x_j} \right),
\end{align*}
\eqref{eq:cauchy} reduces to 
\begin{align}
\rho \left( \frac{\partial \mathbf{u}}{\partial t} + \left( \mathbf{u} \cdot \nabla \right) \mathbf{u} \right) &= - \nabla p + \mu \nabla^2 \mathbf{u} + \rho \mathbf{g} \quad \text{or,} \nonumber \\
\frac{\partial \mathbf{u}}{\partial t} + \left( \mathbf{u} \cdot \nabla \right) \mathbf{u} &= - \frac{1}{\rho} \nabla p + \nu \nabla^2 \mathbf{u} + \mathbf{g}, \label{eq:NSE}
\end{align}
which along with the incompressibility condition, \eqref{eq:incomp}, are the Navier-Stokes equations. Notice that if $\nu = 0$, there is no interaction between fluid elements, and this reduces simply to Euler's equations, \eqref{eq:euler}. 

\subsection{Reynolds-Averaged Navier-Stokes (RANS) Equation}
The Navier-Stokes equations are notoriously difficult to solve since it is a second order partial differential equation, with three components, and four dependencies each. To make solving the Navier-Stokes equations easier, especially for turbulent flows, we can assume the flow is the superposition of the steady, and turbulent flows, and then take the time average. We can make the following variable substitutions:
\begin{align*}
\alpha_i &\equiv \overline{\alpha_i} + \alpha_i', \\
\text{where } \overline{\alpha_i} &= \frac{1}{\tau} \int_0^\tau \alpha_i dt
\end{align*}
is the steady time average on some time scale $\tau$, and $\alpha_i'$ is the turbulent perturbations. It follows from this definition that
\begin{align}
\overline{\alpha_i'} = 0, \quad \frac{\partial \overline{\alpha_i}}{\partial t} = 0, \quad \overline{\overline{\alpha_i}} = \overline{\alpha_i}, \quad \text{and} \quad \overline{\frac{\partial \alpha_i}{\partial x_j}} =  \frac{\partial \overline{\alpha_i}}{\partial x_j}. \label{eq:RANSprop}
\end{align}
We can write \eqref{eq:NSE} in the more suggestive component form:
\begin{align*}
\sum_j \frac{\partial u_i}{\partial t} + u_j \frac{\partial u_i}{\partial x_j} &= \sum_j  -\frac{1}{\rho} \frac{\partial p}{\partial x_i} + \nu \frac{\partial^2 u_i}{\partial x_j^2} + g_i,
\end{align*}
then, making the appropriate substitutions, taking the time average, and using the properties from  \eqref{eq:RANSprop}, we obtain
\begin{align*}
\sum_j \overline{u_j} \frac{\partial \overline{u_i}}{\partial x_j} + \overline{u_j' \frac{\partial u_i'}{\partial x_j}} &= \sum_j  -\frac{1}{\rho} \frac{\partial \overline{p}}{\partial x_i} + \nu \frac{\partial^2 \overline{u_i}}{\partial x_j^2} + \overline{g_i}.
\end{align*}
Rewriting this in vector form gives the Reynolds-Averaged Navier-Stokes equation,
\begin{align}
\left( \mathbf{\overline{u}} \cdot \nabla \right) \mathbf{\overline{u}} + \overline{\left( \mathbf{u'} \cdot \nabla \right) \mathbf{u'}} &= - \frac{1}{\rho} \nabla \overline{p} + \nu \nabla^2 \mathbf{\overline{u}} + \mathbf{\overline{g}}. \label{eq:RANS}
\end{align}

\subsection{Spalart-Allmaras (SA) Turbulent Model}
While \eqref{eq:RANS} makes fluid flow predictions easier, a model is still needed for $\nu$. According to the Boussinesq hypothesis \cite{Boussinesq}, an increase in the viscosity gives the effect of turbulence, and therefore $\nu = \overline{\nu} + \nu'$. One such way to model this altered viscosity and the turbulence, is the Spalart-Allmaras turbulence model. This model determines the turbulent viscosity using the following:
\begin{align}
\nu' = \hat{\nu} f_{v1}; \qquad f_{v1} = \frac{\chi^3}{\chi^3+c_{v1}^3}; \qquad \chi \equiv \frac{\hat{\nu}}{\overline{\nu}}.
\label{eq:saeq}
\end{align}
Assuming the flow is incompressible, and the diffusivity is constant, $\hat{\nu}$ is obtained by solving the convection-diffusion equation,
\begin{align*}
\frac{\partial \hat{\nu}}{\partial t} = D \nabla^2 \hat{\nu} - \overline{\mathbf{u}} \nabla \hat{\nu},
\end{align*}
where $D$ is the diffusivity.

\section{Stanford University Unstructured (SU$^2$) Code}

\begin{wrapfigure}{L}{0.45\textwidth}
\centering
\includegraphics[width = 0.45\textwidth]{Pictures/AirfoilMesh}
\caption[Mesh of an Airfoil]{Unstructured mesh of an airfoil \cite{mesh}.}
\label{fig:airfoilmesh}
\end{wrapfigure}

The Stanford University Unstructured (SU$^2$) code\footnote{\url{http://su2.stanford.edu/}} is a relatively new, open source, constrained partial differential equation solver, optimized for computational fluid dynamics problems. It was initially developed by Francisco Palacios, Thomas Economon, and Juan Alonso from the Aerospace Design Laboratory of the Department of Aeronautics and Astronautics at Stanford University. SU$^2$ is written mostly in C++; and because it is open source, the capabilities have since been extended by many teams of developers from around the world, in addition to the team at Stanford University. \\

SU$^2$ was initially developed under the philosophy that it should remain open source, portable, and highly encapsulated while operating at a high performance standard. The code uses only widely available libraries and dependencies for C++ and Python; furthermore, it is written at a very high, abstract level to minimize disc space, and maximize performance. This philosophy was used to eliminate many of the problems other computational fluid dynamics software are plagued with such as price, compatibility, and memory  \cite{su2main}. Some of the features of the SU$^2$ code include, but are not limited to:
\begin{itemize}
\item Reynolds-Averaged Navier-Stokes equations \eqref{eq:RANS}
\item Spalart-Allmaras Turbulence model \eqref{eq:saeq} ($c_{v1} = 7.1$ \cite{su2main} \cite{su2two})
\item Menter Shear Stress Transport model (an alternative to the SA model)
\item Compressible flows
\item Wave equation, heat equation, and plasma equations
\item Optimal shape design (i.e.\ Supersonic aircraft)
\item Dynamic meshes \\
\end{itemize}

To conduct a simulation, two main files are required, the configuration file, and the mesh file. All of the parameters such as the fluid speed, Reynolds number, and governing equations are specified in the configuration file. The shape geometry of the problem is stored as an unstructured mesh -- hence the name -- in the mesh file, an example visualization is in Figure \ref{fig:airfoilmesh}. The mesh is unstructured in the sense that there is no algebraic relation between nodes. Instead, the mesh is specified by listing the node coordinates, and the adjacent nodes it is connected to. The density of nodes is greatest at the surface since it is the most complex area, and any errors could quickly propagate through the solution. \\

Typically a particular node within the mesh will be surrounded by six triangular cells. Using the centroids of the cells, as well as the midpoints of the edges, we can construct a hexagonal region, $\Omega_i$, which surrounds the $i$th node. Our partial differential equations can be discretized by 
\begin{equation*}
\int_{\Omega_i} \frac{\partial \mathbf{u}}{\partial t} d\Omega + R_i(\mathbf{u}) = 0,
\end{equation*}
where $R_i(\mathbf{u})$ is the residual. After the spacial components have been discretized, the governing equations can be numerically solved with a variety of methods and optimizations. Furthermore, due to the way the solution is discretized, the mesh can dynamically be updated; this is a key feature in the shape optimization process.

\section{Scripting and Automation}

Throughout this project Wolfram Mathematica 10.1.0 for Linux x86 was used to sift through the vast data files and extract the relevant information to produce the plots. The SU$^2$ code natively writes to the data files in the format used for Tecplot\footnote{\url{http://www.tecplot.com/}}, a visualization and analysis tool for computational fluid dynamics. Since Tecplot was not used, a Wolfram function was needed to convert the data to a form Mathematica could use. Once the data was in a usable form, the \texttt{ListDensityPlot} function was used to create the images. The function to tidy the data along with the plotting function were combined into a Wolfram script such as Figure \ref{fig:script}, which can be executed from the terminal. To fully automate the image generation, a shell script was written to iterate over each data file.

\begin{figure}[p]
\lstinputlisting[breaklines=true]{MakeContour.m}
\caption[Automated Wolfram Script]{The Wolfram script used to generate Figure \ref{fig:airfoilanimation}. Syntax highlighting modified from \cite{syntax}.}
\label{fig:script}
\end{figure}

\newpage
\section{Results}

\subsection{Airfoil}

\begin{figure}[htbp]
\begin{subfigure}[t]{0.5\linewidth}
\centering
\includegraphics[width=\textwidth]{./Pictures/Airfoil}
\caption{Steady Euler simulation of an airfoil.}
\label{fig:staticairfoil}
\end{subfigure}
\begin{subfigure}[t]{0.5\textwidth}
\centering
\animategraphics[loop, width = \textwidth, autoplay]{12}{Airfoil_Animation_Turb/Airfoil_Turb}{0050}{0074}
\caption{Turbulent Navier-Stokes simulation of a pitching airfoil, with velocity contours.}
\label{fig:airfoilanimation}
\end{subfigure}
\caption[Turbulent Airfoil Animation, and Laminar Airfoil Streamlines]{Results of the airfoil simulations.}
\label{fig:airfoil}
\end{figure}

\begin{wraptable}{R}{0.5\linewidth}
\centering
\begin{tabular*}{.5\textwidth}{@{\extracolsep{\fill}} | l | c | c |}
 \hline
 & 5.0$^{\circ}$ Angle & Pitching \\ \hline
 Method & Euler & Navier-Stokes \\
 Turbulence & None & Spalart-Allmaras \\
 Flow Speed & 270 ms$^{-1}$ & 270 ms$^{-1}$ \\
 Reynolds Number & 0 & 12560000 \\
 Temperature & 300.0 K & 300.0 K \\
 Pressure & 1 atm & N/A \\
 Iterations & 250 & 25 per frame \\ \hline 
\end{tabular*}
\caption[Airfoil Simulation Parameters]{Parameters used in the airfoil simulations. Results shown in Figure \ref{fig:airfoil}.}
\label{tab:airfoil}
\end{wraptable}

A NACA 0012 airfoil is the first tutorial in the Quick Start section of the documentation \cite{mesh}. This series of airfoils was developed by the National Advisory Committee for Aeronautics (NACA); for this particular case 0012 is the code for a symmetric airfoil with a thickness 12\% of its length. As this is the first tutorial it walks the user through the process of configuring, and running SU$^2$ for the first time after it is installed -- a natural place to start. In the first simulation Euler's equations were used to simulate an airfoil during takeoff. The full specifications of both simulations can be found in Table \ref{tab:airfoil}; this simulation took approximately 45 seconds to complete. Figure \ref{fig:staticairfoil} shows the streamlines, and velocity of the air around the airfoil from the data. It is clear from the data that due to the shape of the airfoil, the velocity of the air that flows over the wing of an aircraft is greatly increased compared to the air that flows under. A consequence of this increase in velocity is a decrease in pressure, which in turn provides lift for the aircraft. \\

The second simulation was of a pitching airfoil, the angle continuously varied from $-4.0^{\circ}$ to $4.0^{\circ}$ -- unlike the first simulation where the angle was a constant $5.0^{\circ}$. Furthermore, the Navier-Stokes equations with turbulence were used. Figure \ref{fig:airfoilanimation} shows the results of the simulation; using a viscous turbulent flow, such as real aircraft experience, there is  undoubtedly a long turbulent stream off of the tail end of the airfoil. Another interesting property is that some of the velocity contours initially form in two distinct areas, and the come together as the airfoil flattens out.

\subsection{Steady Flow Around a Static Cylinder}

boundary layer \\
\lipsum[1]

\begin{figure}[htbp]
\centering
\begin{subfigure}[t]{0.5\linewidth}
  \centering
  \includegraphics[width=0.9\textwidth]{./Pictures/Static_Cylinder}
  \caption{Euler}
  \label{fig:staticeuler}
\end{subfigure}%
\begin{subfigure}[t]{0.5\linewidth}
  \centering
  \includegraphics[width=0.9\textwidth]{./Pictures/Static_Cylinder_NS}
  \caption{NS}
  \label{fig:staticNS}
\end{subfigure} \\
\begin{subfigure}[t]{0.5\linewidth}
 \centering
 \includegraphics[width=0.9\textwidth]{./Pictures/Static_Cylinder_Boundary}
 \caption{Boundaries are slippery.}
 \label{fig:eulerboundary}
\end{subfigure}%
\begin{subfigure}[t]{0.5\linewidth}
 \centering
 \includegraphics[width=0.9\textwidth]{./Pictures/Static_Cylinder_Boundary_NS}
 \caption{The boundary layer that forms due to a viscous fluid.}
 \label{fig:nsboundary}
\end{subfigure}
\caption[Static Cylinder]{Static Cylinders}
\label{fig:static}
\end{figure}

\begin{wraptable}{R}{0.5\linewidth}
\centering
\begin{tabular*}{.5\textwidth}{@{\extracolsep{\fill}} | l | c | c |}
 \hline
 Method & Euler & Navier-Stokes \\ \hline
 Turbulence & None & None \\
 Flow Speed & 75 ms$^{-1}$ & 75 ms$^{-1}$ \\
 Reynolds Number & 0 & 20000 \\
 Temperature & 300.0 K & 300.0 K \\
 Pressure & 1 atm & N/A \\
 Iterations & 250 & 250 \\ \hline
\end{tabular*}
\caption[Non-Turbulent Static Cylinder Simulation Parameters]{Parameters used in the non-turbulent flow simulations. Results shown in Figure \ref{fig:static}.}
\label{tab:static}
\vspace{\baselineskip}
\begin{tabular*}{.5\textwidth}{@{\extracolsep{\fill}} | l | c |}
 \hline
 Method & Navier-Stokes \\ \hline
 Turbulence & Spalart-Allmaras \\
 Flow Speed & 75 ms$^{-1}$ \\
 Reynolds Number & 20000 \\
 Temperature & 300.0 K \\
 Time Step & 0.0006 s \\
 Iterations per Time Step & 10 \\
 Total Iterations & 2000 \\
 Total Time & 1.200 s \\ \hline
\end{tabular*}
\caption[Turbulent Static Cylinder Simulation Parameters]{Parameters used in the turbulent flow simulations. Results shown in Figures \ref{fig:vortexanimation}, and \ref{fig:vortexstill}.}
\label{tab:static}
\end{wraptable}

\subsection{Turbulent Cylinder -- Vortex Shedding}

The equation \eqref{eq:euler} is interesting.

\begin{figure}[htbp]
\centering
\animategraphics[label=vortex, loop, width = \textwidth, autoplay]{12}{Vortex_Animation/Vortex_Shedding}{00}{26}
\caption[Vortex Shedding Animation]{Neat! The full video can be found at \url{https://youtu.be/1lBQNlEdXss}}
\label{fig:vortexanimation}
\end{figure}


\lipsum[1-2]

\begin{figure}
\begin{subfigure}{\linewidth}
  \centering
  \includegraphics[height=0.17\paperheight]{./Pictures/Vortex_Shedding1600}
  \caption{$t = 0$}
\end{subfigure} \\
\begin{subfigure}{\textwidth}
  \centering
  \includegraphics[height=0.17\paperheight]{./Pictures/Vortex_Shedding1628}
  \caption{$t = \frac{1}{4}T$}
\end{subfigure} \\
\begin{subfigure}{\linewidth}
  \centering
  \includegraphics[height=0.17\paperheight]{./Pictures/Vortex_Shedding1656}
  \caption{$t = \frac{1}{2}T$}
\end{subfigure} \\
\begin{subfigure}{\textwidth}
  \centering
  \includegraphics[height=0.17\paperheight]{./Pictures/Vortex_Shedding1684}
  \caption{$t = \frac{3}{4}T$}
\end{subfigure}
\caption[Vortex Shedding at $\frac{1}{4}T$ increments]{Vortex shedding!}
\label{fig:vortexstill}
\end{figure}

\section{Conclusion}

\lipsum[5-7]

\newpage
\begin{thebibliography}{9}

\bibitem{su2main} %Main SU2 paper used for reference
Francisco Palacios, Juan Alonso, Karthikeyan Duraisamy, Michael Colonno, Jason Hicken, Aniket Aranake, Alejandro Campos, Sean Copeland, Thomas Economon, Amrita Lonkar, Trent Lukaczyk, and Thomas Taylor, \emph{Stanford University Unstructured (SU$^2$): An open-source integrated computational environment for multi-physics simulation and design}, 51st AIAA Aerospace Sciences Meeting including the New Horizons Forum and Aerospace Exposition, 2013.

\bibitem{su2two} %The other paper
Francisco Palacios, Thomas D. Economon, Aniket Aranake, Sean R. Copeland, Amrita K. Lonkar, Trent W. Lukaczyk, David E. Manosalvas, Kedar R. Naik, Santiago Padron, Brendan Tracey, Anil Variyar, and Juan J. Alonso, \emph{Stanford University Unstructured (SU$^2$): Open-source Analysis and Design Technology for Turbulent Flows}, 52nd Aerospace Sciences Meeting. National Harbor, Maryland, 2014.

\bibitem{textbook}
John D. Anderson, Jr, \emph{Computational Fluid Dynamics}, McGraw-Hill, Inc., 1995.

\bibitem{Boussinesq} %Increased viscosity -> turbulence
D.C. Wilcox, \emph{Turbulence Modeling for CFD}. 2nd Ed., DCWIndustries, Inc., 1998.

\bibitem{syntax} %Mathematica syntax highlighting
Belisarius (\url{http://mathematica.stackexchange.com/users/29581/belisarius}), \emph{Highlighting Mathematica code in \LaTeX document}, URL: \url{http://mathematica.stackexchange.com/a/83811}

\bibitem{figtab} %List of figures and tables in one place
yo' (\url{http://tex.stackexchange.com/users/11002/yo}), \emph{How to make a combined “List of Figures and Tables”?}, URL: \url{http://tex.stackexchange.com/a/258508}

\bibitem{mesh} %Image of mesh
\url{https://github.com/su2code/SU2/wiki/Quick-Start}

\bibitem{aeroplane} %Image of aeroplane CFD simulation
\url{http://news.stanford.edu/news/2012/january/aero-engineering-software-012412.html}

\end{thebibliography}

\end{document}