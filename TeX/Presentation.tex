\documentclass{beamer}
\usepackage[english]{babel}
\usepackage{amsmath}
\usepackage{color}
\usepackage{amsfonts}
\usepackage{xcolor,graphicx}
\usepackage{geometry}
\usepackage{animate}
\usepackage{listings}
\usepackage{subcaption}
\usepackage{hyperref}

\title{Computational Fluid Dynamics}
\author{Brady Metherall \and 100516905}
\date{Thursday April 7, 2016}

\usetheme{Berkeley}
\usecolortheme{whale}

\begin{document}
\frame{\titlepage}
\setlength\parindent{0pt}

\section{Introduction}

\frame{
\frametitle{Introduction}

}

\section{Theory}

\frame{
\frametitle{Non-Turbulent Flow}

}

\frame{
\frametitle{Reynold's-Averaged Navier-Stokes}

}

\frame{
\frametitle{Spalart-Allmaras Turbulence}

}

\section{$SU^2$ Code}

\frame{
\frametitle{$SU^2$ Code}

}

\frame{
\frametitle{Mesh and Numerics}

}

\section{Scripting and Automation}

\frame{
\frametitle{Scripting and Automation}

Throughout this project Wolfram Mathematica 10.1.0 for Linux x86 was used to sift through the vast data files and extract the relevant information to produce the plots. The $SU^2$ code natively writes to the data files in the format used for Tecplot, a visualization and analysis tool for computational fluid dynamics \cite{tecplot}. Since Tecplot was not used, a Wolfram function was needed to convert the data to a form Mathematica could use. Once the data was in a usable form, the \texttt{ListDensityPlot} function was used to create the images. The function to tidy the data along with the plotting function were combined into a Wolfram script such as Figure \ref{fig:script}, which can be executed from the terminal. To fully automate the image generation, a shell script was written to iterate over each data file.

}

\section{Results}

\frame{
\frametitle{Airfoil}

}

\frame{
\frametitle{Static Cylinder}

}

\frame{
\frametitle{Vortex Shedding}

}

\section{Conclusion}

\frame{
\frametitle{Conclusion}

}

\end{document}